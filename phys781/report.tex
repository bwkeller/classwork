\documentclass[12pt]{article}
\usepackage{amssymb}
\usepackage{graphicx}
\usepackage{setspace}
\usepackage{geometry}
\usepackage{gensymb}
%\usepackage{epsfig}
\linespread{1.9}

\geometry{
  body={5in, 8.5in},
left=1.25in,
right=1.25in,
  top=1.5in
}


\begin{document}

\title{Physics 781 Term Report:\\  Bulges, Black Holes, and Cores:\\ A Review of
the $M_{BH}-\sigma$ and  $M_{BH}-M_{bulge}$ Relation in Theory \& Observation}
\author{Ben Keller}
\maketitle
\newpage

\section{Introduction}
when one examines the structure and evolution of a galaxy, at first glance it 
would seem that the central supermassive black hole (SMBH) would play a 
minimal role in the larger observable properties of the galaxy.  On both mass 
and length scales, the SMBH is small compared to the galaxy as a whole.  For
a typical Milky-Way type galaxy, the SMBH mass is 4 orders of magnitude less 
than the total baryonic mass of the galaxy 
($4*10^6M_\odot$ vs $6*10^{10}M\odot$).  The Schwartzchild radius of this black
hole is even further from the scale of the galaxy, falling 11 orders of 
magnitude smaller than the disk radius ($4*10^{-7}$pc vs. $25$kpc).  Recent 
developments have shown, however, for the spheroidal bulge component of disk 
galaxies, the mass of this SMBH has a tight correllation with the large scale
properties of the bulge, most notably the total bulge mass and the velocity 
dispersion of stars within the disk.  Upon a more thorough examination of the
physical processes governing this system, and the history of its evolution, it
should become clear to the reader that this is neither particularly mysterious
nor is it truly something that should be unexpected.  Despite this, there still
remains some uncertainty regarding the precise nature of these (hence
refered to as the $M_{BH}-M_{bulge}$ and $M_{BH}-\sigma$) relations.  In this
paper, I hope to guide the reader to an understanding of the theory behind these
observations, as well as the current state of knowledge on the topic.

\section{Background \& Observations}
Simply by looking at night sky from a southern location, one can deduce that
the Milky Way galaxy is densest in a region lying in the constellation 
Saggitarius.  Kapteyn was able to determine, roughly a century ago, that the
shape of our galaxy was that of a disk, and that our sun roughly halway along
the radius of this disk.  Close examination of the core region, Sgr A, has 
until relatively recently, been difficult due optically thick obscuring dust.
With the advent of long wavelength astronomy, it has become possible to directly
image stars and gas within the galactic core.  It has been known from infrared
observations since Genzel \& Townes 1987 that the core radius of the galaxy is
extremely compact, with current estimates placing it at $\approx 0.5$pc in
radius (Merrit 2010).  Warm gas and dust have also been observed in infrared and
radio wavelengths at higher densities in this core region.

\subsection{A Central, Supermassive Black Hole}
Within the core region of Sgr A exists a bright radio and hard X-ray emitter, 
Sgr A*.  X-ray observations of this source have constrained it to being highly
compact, and extremely hot ($10^8-10^{10}$K).  This observation points towards
a highly energetic source within the core.  This source explained originally by
two competing models: a driving SMBH, heating gas through accretion, or a 
particularly violent and compact region of starburst activity.  The X-ray 
emission observed within this core region constrained the mass of the black hole
only very loosely ($M_{BH} > 100 M_\odot$) if starburst activity and UV heating
of the gas was the source of the gas heating.  The starburst model truly fails
in the face of one important datum:  the enclosed mass of the core.  By 
examining the rotation curve of stars and gas very close to the core, the 
mass enclosed by that core was able to be determined, as is shown in figure NUM.
As is clear from the figure, the SMBH model is a perfect fit for the observed 
rotation curve.  In addition to this, the densities for the core are so large
that there is no configuration of stars that could be long lived at that 
density.  Currently, the SMBH hypothesis is the consensus view of the dark mass
at the galactic center.

The SMBH hypothesis explaining the central dark mass of the Milky Way was an
attractive answer for another group of astronomers working on a separate 
question:  quasars.  The extremely high luminosity (on the order of 
$10^{13}L\odot$) of these objects was a
mystery for a number of years, and hinted at a driving engine far more energetic
than any stellar effects could account for.  As the physics of black hole
accretion developed, it was realized that accretion onto a SMBH could easily 
produce the luminosity of quasars.  As quasars were hypothesized as the early
stages of modern disk galaxies, the existence of an SMBH in the Milky Way was
taken as further confirmation of the SMBH accretion model of quasar emission.
Further evidence of both the quasar connection and the SMBH hypothesis was found
in the form of radio variability in the form of rapid flares in radio intensity
coming from Sgr A*, corresponding to the accretion of stars or dense clouds.

\subsection{Extragalactic SMBHs}
Regardless of what the Copernican principle might tell us, the existence of a
SMBH within the Milky Way is insufficient evidence for any sort of universal
relationship between disk galaxies and SMBHs.  In the decades following the 
discovery of the Milky Way SMBH, gas and stellar dynamical observations were 
made of nearby disk galaxies, and in a number of these galaxies, similar mass
distributions were found to that of our own galaxy.  Both gas and stellar 
velocity distributions in the inner few parsecs of numerous disk
galaxies, including our two nearest disk neighbours M31 and M33, suggested the
presence of a SMBH within them.  It is now believed that the majority of disk
galaxies contain a SMBH at their core (Magorrian et al. 1997 places the fraction
at 97\% for early-type galaxies).

\subsection{The $M_{BH}-M_{bulge}$ and $M_{BH}-\sigma$ Relations}
We are used to seeing relations between masses and velocity distributions before
in the form of the Tully-Fisher and Faber-Jackson relations for galaxies (though
in these cases, it is an indirect measure of stellar mass through their 
luminosity).  This should not come as much of a suprise to us, since the 
Virial Theorem tells us there must be a relation between the kinetic energy and
the enclosed mass.  In the case of galactic spheroids, a similar relation is 
observed between the total bulge mass, the stellar velocity dispersion $\sigma$
within the bulge, and the mass of the black hole. By measuring the mass of the
central dark object (the SMBH) through velocity dispersions at very small 
radii and the mass of the bulge through dispersions at larger radii, a number
of studies (Magorrian et al. 1997, Ford et al. 1997, and others) found that the
SMBH and bulge masses were fairly well correlated, with a $M_{BH}$ between
$2*10^{-3}M_{bulge}$ and $6*10^{-3}M_{bulge}$.  Observers, as one might imagine,
also found a correlation between 

\section{Theory}
While observations have done an excellent job of showing the \textit{existence}
of the $M_{BH}-\sigma$ relation, they leave open the obvious question as to the
nature of this relation, namely how and why it arises.  The first major attempt
to answer this question was presented in Silk \& Rees 1998, with a coevolution 
model.  This model starts from rather simple assumptions.  First, they assume
that some primoridal gas clouds collapse to form 
\section{Recent Developments}
\section{Conclusion}
\section*{References}
\begin{enumerate}
\item \textit{Exploring the unusually high black hole-to-bulge mass ratios in 
NGC4342 and NGC4291: the asynchronous growth of bulges and black holes} Bogdan 
et al.,  submitted to ApJ, retrieved via arXiv 2012
\item \textit{The Distribution of Stars and Stellar Remnants at the Galactic 
Center} Merritt, ApJ 718, 2010
\item \textit{The Supermassive Black Hole at the Galactic Center} Melia \& 
Falcke, ARAA 39, 2001
\item \textit{Quasars and Galaxy Formation} Silk \& Rees, A\&A 331, 1998
\item \textit{The Demography of Massive Dark Objects in Galactic Centers} 
Magorrian et al., AJ 115, 1998
\item \textit{Inward Bound -- The Search for Supermassive Black Holes in 
Galaxtic Nuclei} Kormendy \& Richstone, ARAA 33, 1995
\item \textit{Physical Conditions, Dynamics, and Mass Distribution in the Center
of the Galaxy} Genzel \& Townes, ARAA 25, 1987
\end{enumerate}
\end{document}
