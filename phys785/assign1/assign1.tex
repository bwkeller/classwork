\documentclass{article}
\author{Ben Keller}
\title{Physics 785 Assignment 1}
\begin{document}
\maketitle
\centering
\section*{A Virialized Filamentary Infrared Dark Cloud}
\subsection*{Abstract}
Infrared Dark Clouds (IRDC) match the initial conditions of massive 
and clustered star formation, and are thus an excellent tool
to probe the earliest stages of these modes of star formation.
Using IRAM molecular line observations of $C^{18}O$ along
with MIREX extinction maps, we measure the kinematics of a
filament containing a few $10^3 M_\odot$, with a length of
$\approx4pc$.  We use this data to obtain the total velocity
dispersion $\sigma$ along the line of sight to approximately
$10\%$ accuracy. This allows us to use a simple theoretical
model of the filament geometry to perform a virial analysis
on the filament probing smaller scales (and produce better
mass estimates) than done in 
previous studies, such as Hernandez and Tan 2011. Our results
show that, contrary to previous studies, IRDC filaments are
in (or near) virial equilibrium.  If IRDC filaments are the
initial locations of star formation, this result implies that
stars form from gas that is in rough pressure equilibrium, 
validating one of the main assumptions of McKee and Tan's
core accretion model of massive star formation.
\end{document}
