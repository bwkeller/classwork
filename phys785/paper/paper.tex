\documentclass[12pt, preprint]{aastex}

\begin{document}

\title{Physics 785 Term Paper: Turbulence in Molecular Clouds: Origins \&
Impacts}
\author{Ben Keller}
\maketitle
\newpage
\section{Introduction}
The phenomenon of turbulence is one of the most heavily studied, and yet poorly
understood processes in physics.  It can influence the flow of fluids across
length scales spanning more than a dozen orders of magnitude, and can affect
astrophysical processes at every scale, from dense cores up to the intergalactic
medium (\citet{scan2013} suggests that cold galactic outflows can be
driven by turbulence in the ISM).  In this paper, I will review the current understanding of turbulence in
the molecular cloud environment.
\section{Basic Turbulence Theory}
\section{Observational Evidence}
\section{Origins of Turbulence}
\section{Impact of Turbulence on Cloud Structure \& Evolution}
\section{Conclusion}
\begin{thebibliography}{}
	\bibitem[Scannapieco(2013)]{scan2013} Scannapieco, E.\ 2013, \apjl, 763, L31 
\end{thebibliography}
\end{document}
