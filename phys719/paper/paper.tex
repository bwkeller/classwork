\documentclass[12pt, preprint]{aastex}
\usepackage{float}
\usepackage[cm]{fullpage}
\begin{document}

\title{Physics 719 Term Paper: AREPO, a Hydrodynamics Solver on an Unstructured
Moving Mesh}
\author{Ben Keller}
\maketitle
\newpage
\section{Introduction}
AREPO solves the Euler equations using a novel method: rather than calculating
fluxes through a fixed, structured grid (Cartesian or otherwise), it generates
an unstructured mesh on the fly from state information of the volume and
uses that mesh to define where fluxes are calculated \citep{springel2009}.  By
allowing the fluid flow to define the grid, AREPO is meant to serve as a
compromise between the alternate formulations of hydrodynamics: Eulerian (i.e.
grid based methods), and Lagrangian (ie, particle based methods).  It is part of
a family of hybrid techniques: ``Arbitrary Lagrangian-Eulerian'' (ALE) methods.

While both the Lagrangian and Eulerian formulations are equally valid from a
mathematical standpoint, each has weaknesses when applied to solve problems
where only finite resolution and computational resources are available. Eulerian
codes are able to use Godunov-type methods to capture shocks without the need
for high spatial resolution, and also suppress spurious artifacts (post-shock
ringing, etc.), while smoothed particle hydrodynamics (SPH, the dominant
Lagrangian method) require both more resolution and carefully chosen explicit
artificial viscosity (AV) to accurately capture strong shocks.  Eulerian codes
suffer from the fact that their grids act as unphysical preferred frames of
reference (they lack Galilean invariance), introducing spurious diffusion as
features are smeared over grid cells. Additionally, they often require much
smaller timesteps criteria than SPH codes.  

In this paper, I will first explain how AREPO solves the Euler equations using a
finite volume method over a moving Voronoi mesh.  A Voronoi mesh is a unique
tessellation of all points where each cell is composed of space closer to the
contained mesh-generating point than any other mesh generation point.  As a
large amount of effort is required to construct and maintain this mesh, the
second section covers these mesh-generation routines.  In the penultimate
section, I explain how AREPO does its time integration, as it uses a careful
timestep criteria and multiple timesteps. Finally, I conclude with
some remarks about where AREPO stands in a somewhat crowded field of
astrophysics simulation codes.  AREPO also includes modifications to the Euler
equations for self-gravitating fluids and cosmological expansion, mesh
refinement and de-refinement, as well as a dual-energy formulation for use in
low-resolution simulations of cold, supersonic flows that I will not discuss
here out of space consideration.
\section{AREPO's Finite Volume Hydrodynamics Method} 
\subsection{Core Method}
While the idea of solving the fluid equations on an unstructured mesh
\citep{xu1997} or a moving mesh \citep{gnedin1995} is not new, using
\textit{both} of these methods simultaneously was not heavily explored prior to
AREPO.  The basic derivation of the method follows fairly similarly to 
other Monotone Upstream-centered Schemes for Conservation Laws (MUSCL), with a
few differences.  To integrate the primitive state vector $\mathbf{U}$, with
flux vector $\mathbf{F}$ and cell-averaged value $\mathbf{Q}$:
\begin{equation}
	\begin{array}{l l l}
		\mathbf{U} = (\rho, \rho\mathbf{v}, \rho e),  &
		\mathbf{F} = (\rho\mathbf{v}, \rho v^2+P, \mathbf{v}(\rho e+P)), &
	\mathbf{Q_i} = (m_i, \mathbf{p_i}, E_i), \\
\end{array}
\end{equation}
AREPO first calculates a flux across across each face (with surface normal
$\mathbf{A_{ij}}$ of a given mesh cell):
\begin{equation}
	\mathbf{F_{ij}} = \frac{1}{A_{ij}}\int_{A_{ij}}
	[\mathbf{F(U)-Uw}]d\mathbf{A_{ij}}
\end{equation}
where $\mathbf{w}$ is the velocity of the the boundary itself (naturally, this
term is omitted in the Eulerian form).  With these fluxes, the cell-averaged
values can be updated simply using a Godunov-type scheme:
\begin{equation}
	Q_{i}^{n+1} = Q_i^n -\Delta t \sum_j A_{ij} \mathbf{\hat F_{ij}^{n+1/2}}
\end{equation}
where $\mathbf{\hat F_{ij}}$ is a time-average approximate flux.  AREPO uses an
unsplit method, where fluxes are calculated for each face before any are
applied.  These fluxes can be generated using a slope-limited piecewise linear
approximation.  To determine the slope of the approximation, AREPO uses a
gradient for each of the primitive variables, estimated using the vector between
the mesh-generating points (more details on these points can be found in section
3), $\mathbf{r_{ij}}$, and the vector from the midpoint of this vector to the
$ij$-face's center of mass $\mathbf{c_{ij}}$:
\begin{equation}
	\nabla \phi_i = \frac{1}{V_i}\sum_{i\neq j}
	A_{ij}\left([\phi_j-\phi_i]\mathbf{\frac{c_{ij}}{r_{ij}}} -
\frac{\phi_i+\phi_j}{2}\frac{\mathbf{r_{ij}}}{r_{ij}}\right)
\end{equation}
The linear approximation for in each cell, with center of mass $\mathbf{s_i}$
becomes: 
\begin{equation}
	\phi(\mathbf{r}) = \phi_i + \nabla \phi_i\cdot(\mathbf{r-s_i})
\end{equation}
This gives a second order space and time accuracy, with a stencil consisting of the
cell and all its neighbours.  In order for the integration scheme
to remain monotonicity preserving (MP), a slope limiter is used to ensure that the
estimate doesn't over/undershoot in cases involving discontinuous flow:
\begin{equation}
	\nabla\phi_{i,limited} = \mathrm{min}(1,\Psi_{ij})\nabla\phi_{i}
\end{equation}
\begin{equation}
	\Psi_{ij} = \left\{
		\begin{array}{l l l}
			(\phi_i^{max}-\phi_i)/\Delta\phi_{ij} & if & \Delta\phi_{ij} > 0\\
			(\phi_i^{min}-\phi_i)/\Delta\phi_{ij} & if & \Delta\phi_{ij} < 0\\
			1 & if \Delta\phi_{ij} = 0
	\end{array} \right.
\end{equation}
This acts to reduce the order of the method near discontinuities, and ensures
that monotonicity is preserved. AREPO uses a somewhat liberal slope limiter 
that makes it MP, but not total variation diminishing (TVD), making the scheme
less dissipative overall at the cost of allowing some artifacts to be produced
at shocks.
\subsection{Calculating Fluxes}
To calculate the flux through a cell face, first the face's local frame velocity
must be calculated.  For a face bounding two cells, with velocities $w_R$ and
$w_L$, the face velocity $w$ is:
\begin{equation}
	w = \frac{w_R+w_L}{2}+w'
\end{equation}
where $w'$ is a rotation term to account for the component of the velocity
arising from the relative motion of the cell centers of mass and their
mesh-generating points.  Once this is calculated, the lab-frame states of each
cell is first boosted to the face's rest frame.  The state vectors are then
interpolated to find their values on the face's centroid, at $\Delta t/2$.
These interpolated values are then multiplied by a rotation matrix $\Lambda$ to bring
their x-axis perpendicular to the face.  This finally allows the solution
$\mathbf{W_F}$ to a 1D Riemann problem across the face to be calculated. By
reversing the rotation and the boost, this solution in the lab frame
$\mathbf{W_{lab}}$ can be determined:
\begin{equation}
	\mathbf{W_{lab}} = \Lambda^{-1}\mathbf{W_F+w}
\end{equation}
This is the used to generate the flux vector $\mathbf{\hat F}$:
\begin{equation}
	\mathbf{\hat F = F(W_{lab}) - W_{lab}\cdot w}
\end{equation}
Using this, the AREPO scheme is Galilean invariant, as fluxes are all
calculated in the invariant boundary frame.
\section{Building \& Maintaining Voronoi Meshes}
Solving the fluid equations over a Voronoi mesh is only half the battle of
evolving a set of initial conditions with AREPO.  Before each step in the
hydrodynamics solver begins, a new tessellation must be computed; after each
step, the points that generate the mesh must be moved (at the velocity of their
cell's fluid).
\subsection{Initial Construction}
Given a set of points, there exists only one tessellation in which each cell
contains only points closest to its contained point: the Voronoi tessellation.
AREPO exploits a convenient fact that every Voronoi tessellation has a dual
tessellation, the Delaunay, that is much easier computed.  The Delaunay mesh is a
triangulation which places a triangle across chords of a circumsphere that contains on
its edge the three vertex points (mesh generating points), such that this
circle contains no mesh generating points.  This can be built using recursive
divide-and-conquer algorithms similar to quicksort (e.g. \textit{quickhull}
\citep{barber1996}, \textit{Dewall}\citep{cignoni1997}). These algorithms have
asymptotic cost comparable to tree-building ($O(N\log N)$ in the average case,
$O(N^2)$ in the worst case).  Once the Delaunay triangulation is completed, the
Voronoi faces are generated using the midpoints of the Delaunay circumspheres as
the vertices of the Voronoi cells.  This gives each cell a set of faces, a
volume, and a center of mass that is then used by the hydrodynamics solver to
evolve the system.
\subsection{Restructuring}
While the ``pure'' Voronoi tessellation is perfectly capable of being evolved
forward in time without introducing error-causing mesh artifacts, it can often
produce a very anisotropic mesh, even in situations where the physical density
of a volume is isotropic. This doesn't affect the accuracy of the method, but it
does increase its cost.  To deal with this, AREPO introduces a small numerical
``kick'' to the mesh-generator velocities, to move their mesh-generating points
towards their centers of mass.  If these points are close, then the resulting
Voronoi cells are highly regular at equal density after a small number of
iterations.  AREPO adds a velocity kick (less than the sound speed $c_i$) to the velocities of the mesh
generators if the distance $d_i$ between their centers of mass $\mathbf{s_i}$
and mesh-generators $\mathbf{r_i}$ become more than some fraction $\eta$ of the the
cell radius $R_i$ (see equation 12).  
\begin{equation}
	\mathbf{w_i'} = \mathbf{w_i}+\left\{
		\begin{array}{l l l}
			0 & if & d_i/(\eta R_i) < 0.9\\
			c_i\frac{\mathbf{s_i-r_i}}{d_i}\frac{d_j-0.9\eta R_i}{0.2\eta R_i} & if & 0.9\leq d_i/(\eta R_i) < 1.1\\
			c_i\frac{\mathbf{s_i-r_i}}{d_i} & if & 1.1\leq d_i/(\eta R_i) \\
	\end{array} \right.
\end{equation}
AREPO typically uses a threshold of $\eta=1/4$. This kick acts to keep cells
``round'' as the simulation evolves.  As the solver is fully Galilean invariant,
the boost resulting from these kicks does not result in any errors introduced in
the hydrodynamics solutions.  It appears to me that this would introduce a small
amount of shot-noise into the mesh velocity field, which may manifest in the
solution AREPO produces.
\section{Time Integration \& Independent Timesteps}
The general CFL condition used in AREPO calculates an effective distance scale
$R_i$, sound speed $c_i$ and flow velocity relative to the grid $v_i'$ (which is
typically negligible because of the semi-Lagrangian mesh motion).  Preventing
$R_i$ from becoming small is one of the primary reasons for mesh restructuring.
\begin{equation}
	\begin{array}{l l l}
		R_i = (3V_i/4\pi)^{1/3}, & &
	\Delta t_i = C_{CFL}\frac{R_i}{c_i+v_i'}\\
\end{array}
\end{equation}
AREPO typically adopts $C_{CFL} \approx 0.4-0.8$.  Since these timesteps are
calculated individually per cell, individual cells are capable of being run at
different time resolutions, depending on their local velocities and densities.
AREPO gives each cell a timestep that is a power of two smaller than the global
timestep, such that the above timestep criteria is obeyed for all cells.

Since this calculation only examines local properties, it needs a way to
anticipate interactions with approaching supersonic flows. In order to do this,
AREPO takes the smaller of the timestep from equation 12 or the distance over
the ``signal speed'' (flow velocity+sound speed) between two cells.  It uses a
tree-based search in order to find all other cells and calculate the non-local
timesteps for each cell (at a cost of $O(N\log N)$, compared to $O(1)$ for a
\citet{saitoh2009} type limit).

For each face, the smaller timestep of the two cells at the interface is used.
Each substep then calculates fluxes for all faces that reside on that timestep
or, a smaller timestep (active faces) (e.g. a face on a step 1/4 the timestep of
another face will calculate fluxes 4 times for every 1 time the long step face
calculates them).  At the end of each substep, the mesh is rebuilt for active
cells.  Cells that are not active ``store up'' fluxes coming from faces that are
on shorter timesteps, and apply them once they complete their step.


This looks to me to be potentially the most problematic aspect of the method, as
you ought to have (if you want to preserve the Lagrangian aspect) the grid fully
rebuilt every time \textit{any} cell is evolved, but the grid is only rebuilt
for the active cells.  This means the faces that the fluxes between cells on
small timesteps and cells on large timesteps become less accurate as the
difference in timestep increases (and thus the fluxes calculated become less
accurate). In other words, the longer the range in timesteps, the less
Lagrangian the method is.

\section{Conclusion}
AREPO seems to be a clever mix of the Eulerian and Lagrangian approaches to
fluid mechanics.  It promises to deliver accurate, low diffusivity shock
capturing with a 2nd order space and time accurate, unsplit Godunov MUSCL
method.  At the same time, it has modified the Eulerian version of this method
to be Galilean invariant, and provide Courant times competitive with SPH.  For
simulations of highly supersonic, cold flows (situations like turbulence in the
cold ISM), AREPO likely offers one of the best methods available.

Whether it becomes a serious challenge to popular SPH and AMR codes will depend
on whether the additional cost of the method is worth the accuracy it provides.
AREPO needs to solve 2 Riemann problems per face, with much more faces per cell
than in a standard Cartesian grid.  Its timestep estimator is also fairly
expensive and quite conservative.  In a fairly small galaxy cluster test ($32^3$
particles), AREPO was 60\% slower than a comparable SPH code.  If AREPO can be
shown to scale well to both large numbers of cells and large numbers of
processors, it has a good chance of becoming one of the frontrunners in the
field of cosmological hydrodynamics codes.
\begin{thebibliography}{}
	\bibitem[Gnedin (1995)]{gnedin1995} Gnedin, N. Y. 1995.
		ApJS, 97, 231
	\bibitem[Bradford Barber et al. (1996)]{barber1996} Bradford Barber, C. et
		al. 1996.
		ACM Transactions on Mathematical Software 22, 469
	\bibitem[Xu (1997)]{xu1997} Xu, G. 1997.
		MNRAS 288, 903
	\bibitem[Cignoni Z et al. (1997)]{cignoni1997} Cignoni Z, P et al. 1997.
		Computer-Aided Design, 30, 333
	\bibitem[Saito \& Makino (2009)]{saitoh2009} Saitoh, T. R. \& Makino, J
		2009. ApJ 697, 99
	\bibitem[Springel (2009)]{springel2009} Springel, V. 2009.
		MNRAS 401, 791
\end{thebibliography}

\end{document}
